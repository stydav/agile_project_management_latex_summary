\section*{glossary}

\begin{table}[H]
  \begin{tabular}{lll}
    \textbf{burndown chart} &  shows the amount of work remaining across time, the correlation between amount of work remaining and the progress of the Team(s) in reducing it\\
    \textbf{'chicken'} &  someone in the project which role implies that he is not 'on the hook'\\
    \textbf{defined process control} &  Laying out a process that repeatably will produce acceptable quality output\\
    \textbf{Daily Scrum} & team gets together for a 15-minute meeting\\
    \textbf{empirical process control} &  When defined process control cannot be achieved because of the complexity of the intermediate activities\\
    \textbf{functional /non-functional requirements} &  While functional requirements define what the system does or must not do, non-functional requirements specify how the system should do it\\
    \textbf{management roles} & Product Owner, ScrumMaster, and Team (all three 'pigs')\\
    \textbf{product Backlog} & The list of functional and nonfunctionalrequirements\\
    \textbf{'pig'} & someone in the project which role implies that he is 'on the hook'\\
    \textbf{ROI} &  return on investment\\
    \textbf{Scrum of Scrums} &  the work of many teams is coordinated by individuals from each of the teams.\\
    \textbf{stakeholders} &  those with an interest in the software and how it works, who fund the project, those who intend to use the functionality created by the project, and those who will be otherwise affected by the project\\
    \textbf{vision} &  describes why the project is being undertaken and what the desired end state is.\\
   & 
  \end{tabular}
\end{table}