\section*{quotations}


\begin{quotation}
  Scrum hangs all of its practices on an iterative,  process skeleton. The heart of Scrum lies in the iteration.
\end{quotation}


\begin{quotation}
  Scrum addresses the complexity of software development projects by implementing the inspection, adaptation, and visibility requirements of empirical process control with a set of simple practices and rules, which are described in the following sections. 
\end{quotation}


\begin{quotation}
  The people who fill these roles are those who have committed to the project ('pigs'). Others might be interested in the project, but they aren’t on  he hook. Scrum makes a clear distinction between these two groups and ensures that those who are responsible for the project have the authority to do what  s necessary for its success and that those who aren’t responsible can’t interfere unnecesarily ('chickens'). 
\end{quotation}


\begin{quotation}
  All work is done in Sprints
\end{quotation}


\begin{quotation}
  
\end{quotation}


\begin{quotation}
  
\end{quotation}


\begin{quotation}
  
\end{quotation}


\begin{quotation}
  
\end{quotation}


\begin{quotation}
  
\end{quotation}


\begin{quotation}
  
\end{quotation}


\begin{quotation}
  
\end{quotation}


\begin{quotation}
  
\end{quotation}


\begin{quotation}
  
\end{quotation}


\begin{quotation}
  
\end{quotation}


\begin{quotation}
  
\end{quotation}


\begin{quotation}
  
\end{quotation}

