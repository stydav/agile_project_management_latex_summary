\section*{quotations}


\begin{quotation}
  Scrum hangs all of its practices on an iterative,  process skeleton. The heart of Scrum lies in the iteration.
\end{quotation}


\begin{quotation}
  Scrum addresses the complexity of software development projects by implementing the inspection, adaptation, and visibility requirements of empirical process control with a set of simple practices and rules, which are described in the following sections. 
\end{quotation}


\begin{quotation}
  The people who fill these roles are those who have committed to the project ('pigs'). Others might be interested in the project, but they aren’t on  he hook. Scrum makes a clear distinction between these two groups and ensures that those who are responsible for the project have the authority to do what  s necessary for its success and that those who aren’t responsible can’t interfere unnecesarily ('chickens'). 
\end{quotation}


\begin{quotation}
  Defer from building an inventory of Product Backlog until you are ready to engage a Team to convert it to functionality.
\end{quotation}


\begin{quotation}
  Each takes should take roughly 4 to 16 hours to finish, tasks taking longer than 4 to 16 hours are considered mere placeholders for tasks.
\end{quotation}


\begin{quotation}
  Scrum requires Teams to build an increment of product functionality every Sprint
\end{quotation}

\begin{quotation}
  'done" means that code is thoroughly tested, well written and the user functionality is documented
\end{quotation}

\begin{quotation}
  Scrum practices regularly make visible a project’s progress, problems, and sociology.
\end{quotation}


\begin{quotation}
  The Scrum process defines practices, meetings, artifacts, and terminology.
\end{quotation}


\begin{quotation}
  Traditionally, customers get to state the requirements that optimize their ROI at the start of the project, but they don’t get to assess the accuracy of their predictions until the project is completed. Scrum lets the Product Owner adjust the ROI much more frequently.
\end{quotation}


\begin{quotation}
  The team decides how to turn the selected requirements into an increment of potentially shippable product functionality.
\end{quotation}


\begin{quotation}
  Scrum is structured to regularly make the state of the project visible to the three managers—the Product Owner, the ScrumMaster, and the Team—so that they can rapidly adjust the project to best meet its goals.
\end{quotation}


\begin{quotation}
  Scrum involves a paradigm shift from control to empowerment, from contracts to collaboration, and from documentation to code.
\end{quotation}


\begin{quotation}
  The Product Backlog and its prioritization are open to everyone so that they can discuss them and come to the best way to optimize ROI.
\end{quotation}


\begin{quotation}
  Scrum is an empirical process. Rather than following outdated scripts, Scrum employs frequent inspection and adaptation to direct team activities toward a desired goal.
\end{quotation}


\begin{quotation}
  Scrum teams are cross-functional.
\end{quotation}


\begin{quotation}
  Scrum relies on individual and team commitments rather than on top-down control through planning.
\end{quotation}


\begin{quotation}
  Scrum must be put into place before it can be fully understood.
\end{quotation}


\begin{quotation}
  A primary tool the ScrumMaster can use to improve customer involvement is the delivery of quick results that customers can potentially use in their organization.
\end{quotation}



\begin{quotation}
  
\end{quotation}


\begin{quotation}
  
\end{quotation}


\begin{quotation}
  
\end{quotation}


\begin{quotation}
  
\end{quotation}


\begin{quotation}
  
\end{quotation}


\begin{quotation}
  
\end{quotation}


\begin{quotation}
  
\end{quotation}


\begin{quotation}
  
\end{quotation}


\begin{quotation}
  
\end{quotation}


\begin{quotation}
  
\end{quotation}



\begin{quotation}
  
\end{quotation}


\begin{quotation}
  
\end{quotation}


\begin{quotation}
  
\end{quotation}


\begin{quotation}
  
\end{quotation}


\begin{quotation}
  
\end{quotation}


\begin{quotation}
  
\end{quotation}


\begin{quotation}
  
\end{quotation}


\begin{quotation}
  
\end{quotation}


\begin{quotation}
  
\end{quotation}


\begin{quotation}
  
\end{quotation}

