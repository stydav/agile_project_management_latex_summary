\pagebreak
\chapter{New Management Responsibilities}

all three management roles are 'pig' roles: Product Owner, ScrumMaster, and Team. All other
managers in an organization are 'chickens', they all have to work through the pigs.
Chickens have no direct authority over the project’s execution or progress. Scrum management is responsible for inspecting the aspects of the project that Scrum makes visible and adapting accordingly.

Developers need to focus on producing the product. 

Sales and marketing want quick responses to every opportunity that comes knocking.

Scrum helps to balance the needs of both the marketing and the development departments.

Scrum banns interference during the Sprint, everyone else must leave the developers alone to work $\rightarrow$ redirected to higher priority work at the Sprint planning meeting.

Traditionally, customers get to state the requirements that optimize their ROI at the start of the project, but they don’t get to assess the accuracy of their predictions until the project is completed. Scrum lets the Product Owner adjust the ROI much more frequently.

In a huge reversal of ordinary management practices, Scrum makes the team responsible for managing development activities. Traditionally, the project manager tells the team what to do and manages its work.

The pressure inherent in a 30-day Sprint, the commitment the team members make to each other to accomplish something, and the principles of selforganization and cross-functional responsibilities all help the team successfully fulfill this responsibility.

When anyone outside the team tries to tell the team what to do, more damage than good usually results.

Teams that are too big dont work effectively, a team should optimally have 7 members.


Scrum is structured to regularly make the state of the project visible to the three managers—the Product Owner, the ScrumMaster, and the Team—so that they can rapidly adjust the project to best meet its goals.




