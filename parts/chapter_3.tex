\pagebreak
\chapter{the scrum master}
\label{sec:the_scrum_master}

The ScrumMaster is responsible for the success of the project, and he or she helps increase the probability of success by helping the Product Owner select the most valuable Product Backlog and by helping the Team turn that backlog into functionality. The ScrumMaster earns no awards or medals because the ScrumMaster is only a facilitator. 

Scrum involves a paradigm shift from control to empowerment, from contracts to collaboration, and from documentation to code.

People often hire consultants because they want to get a different perspective on their situations. This new perspective is often perceived as somehow better than the native view of things.


\subsection*{The shift from project manager to ScrumMaster}
\begin{itemize}
  \item shift from controlling to facilitating
  \item from bossing to coaching
  \item from manager to leader
  \item from authority to no authority
  \item from delegating to being personally responsible
  \item from controlling to guiding
\end{itemize}

The ScrumMaster’s job is to protect the team from impediments during the Sprint. However, the ScrumMaster has to operate within the culture of the organization. The ScrumMaster walks a fine line between the organization’s need to make changes as quickly as possible and its limited tolerance for change. 

Whenever possible, the ScrumMaster makes a  case and pushes the necessary changes through. The results are often greater productivity and greater return on investment (ROI). However, sometimes these changes are culturally unacceptable and the ScrumMaster must acquiesce.

Whenever an opportunity arises that was more important than the work selected by the team for the Sprint, management can abnormally terminate the Sprint. The Team, the Product Owner, and management would then conduct a new Sprint planning meeting. The new opportunity would be selected if it truly was the top-priority Product Backlog. Scrum keeps everything highly \textbf{visible}. By keeping everything in full view, the type of backroom politicking and influence swapping normal in most organizations is minimized.


The responsibilities of the ScrumMasters can be summarized as follows:
\begin{itemize}
  \item Remove the barriers between development and the Product Ownerso that the Product Owner directly drives development.
  \item Teach the Product Owner how to maximize ROI and meet his or herobjectives through Scrum.
  \item Improve the lives of the development team by facilitating creativityand empowerment.
  \item Improve the productivity of the development team in any way possible.
  \item Improve the engineering practices and tools so that each incrementof functionality is potentially shippable.
  \item Keep information about the team’s progress up-to-date and visible toall parties.
\end{itemize}

These responsibilities should be enough to keep the ScrumMaster busy; no ScrumMaster should have any time left over to act like a typical boss. ScrumMaster who acts like a program manager probably isn’t fulfilling all of his or her  duties as a ScrumMaster.




