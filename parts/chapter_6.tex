\pagebreak
\chapter{Planning a Scrum Project}

The Scrum planning process sets stakeholders’ expectations. These stakeholders include those who fund the project, those who intend to use the functionality created by the project, and those who will be otherwise affected by the project. The plan is a way of synchronizing stakeholders’ expectations with the Team’s expectations.

At the end of the Sprint, the stakeholders attend the Sprint review meetings and compare the project’s actual progress against its planned progress.

Changes in course and revisions to the plan made in Sprint planning meetings are explained to the stakeholders.

For those who are unable to attend the Sprint review meeting, the project reports compare actual results to the plan—both the original plan and the plan as it has been modified since the project’s inception.

The Scrum planning process involves resolving three questions:

\begin{itemize}
  \item What can those funding the project expect to have changed when the project is finished?
  \item What progress will have been made by the end of each Sprint?
  \item Why should those being asked to fund the project believe that the project is a valuable investment, and why should they believe that those proposing the project can deliver those predicted benefits?
\end{itemize}


Scrum projects require less planning than typical Gantt chart–based projects because those working to deliver the expected benefits provide visibility into their progress at the end of every Sprint.


The minimum plan necesary to start a Scrum project consists of a vision and a Product Backlog.