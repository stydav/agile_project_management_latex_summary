\pagebreak
\chapter{Planning a Scrum Project}

The Scrum planning process sets stakeholders’ expectations. These stakeholders include those who fund the project, those who intend to use the functionality created by the project, and those who will be otherwise affected by the project. The plan is a way of synchronizing stakeholders’ expectations with the Team’s expectations.

At the end of the Sprint, the stakeholders attend the Sprint review meetings and compare the project’s actual progress against its planned progress.

Changes in course and revisions to the plan made in Sprint planning meetings are explained to the stakeholders.

For those who are unable to attend the Sprint review meeting, the project reports compare actual results to the plan—both the original plan and the plan as it has been modified since the project’s inception.

The Scrum planning process involves resolving three questions:

\begin{itemize}
  \item What can those funding the project expect to have changed when the project is finished?
  \item What progress will have been made by the end of each Sprint?
  \item Why should those being asked to fund the project believe that the project is a valuable investment, and why should they believe that those proposing the project can deliver those predicted benefits?
\end{itemize}


Scrum projects require less planning than typical Gantt chart–based projects because those working to deliver the expected benefits provide visibility into their progress at the end of every Sprint.


The minimum plan necesary to start a Scrum project consists of a vision and a Product Backlog.

The vision describes why the project is being undertaken and what the desired end state is. For a system used internally within an organization, the vision might describe how the business operation will be different when the system is installed. For software that is being developed for external sale, the vision might describe the software’s major new features and functions, how they will benefit customers, and what the anticipated impact on the marketplace will be.

The Product Backlog defines the functional and nonfunctional requirements that the system should meet to deliver the vision, prioritized and estimated. It is parsed into potential sprints.





\subsection{implementing scrum in running projects}

Scrum is often implemented well after the project in question has been planned. In the case of these projects, the funding is already in place and expectations have already been established. What’s necessary  ow is to replan the project in light of Scrum so that the Team, Product Owner, and stakeholders can envision the project as a series of Sprints that lead to a release, all driven by the Product Backlog.


The first task is to create the Scrum artifact needed for managing a Scrum project: the Product Backlog.

\subsection{Estimating the Product Backlog}

To estimate each requirement precisely, one would have to know the exact composition and interaction of the requirement, the technology used to build the requirement, and the skills and mood of the people doing the work. One could potentially spend more time trying to define these attributes and their interactions than we would spend actually transforming the requirement into functionality.

The nature of complex problems is such that very small variations in any aspect of the problem can cause extremely large and unpredictable variations in how the problem manifests itself.

The purpose of estimating is to get a handle on the size of each requirement, both in its own right and relative to the size of the other requirements.

This information helps prioritize the Product Backlog and divide it into Sprints.

Scrum is empirical and ultimately based on the “art of the possible.” The team only has to do its best during each Sprint, and the expectations about what could be done are upated by the end of each Sprint.
Actual progress is being tracked on each sprint's product backlog. At the end of every Sprint, expectations are updated by tracking actual delivery of functionality against expected delivery of functionality.

estimates should include
\begin{itemize}
  \item how long it would take to analyze, design, and code the requirements
  \item time for unit testing, test automation
  \item time for code reviews, for refactoring, for writing code cleanly and legibly, and for removing unnecessary code
\end{itemize}


\subsection{What Does 'Done' Mean?}

