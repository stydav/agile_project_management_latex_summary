\subsection{empirical process control}

\begin{quotation}
  If the commodity is of such unacceptable
  quality as to be unusable, the rework is too great to make the price acceptable,
  or the cost of unacceptably low yields is too high, we have to turn to and accept
  the higher costs of \textbf{empirical process control}.
  In the long run, making successful
  products the first time using empirical process control turns out to be much
  cheaper than reworking unsuccessful products using defined process control
\end{quotation}

Three legs of empirical process control:
\begin{itemize}
  \item visibility
  \begin{itemize}
    \item aspects of the process that affect the outcome must be visible to those controlling
    the process. Not only must these aspects be visible, but what is visible must
    also be true (certain functionality is labeled 'done'?). It doesn’t matter whether it is visible that this functionality
    is done if no one can agree what the word “done” means.
  \end{itemize}
  \item inspection
  \begin{itemize}
    \item The various aspects of the process must be
    inspected frequently enough that unacceptable variances in the process can be
    detected. Processes are changed by the very act of inspection. The other factor in
    inspection is the inspector, who must possess the skills to assess what he or she
    is inspecting.
  \end{itemize}
  \item adaption
  \begin{itemize}
    \item when process are outside acceptable limits and that the resulting product will be unacceptable, adjustment
    must be made as quickly as possible to minimize further deviation.
  \end{itemize}
\end{itemize}

'code review' as an example of an empirical process control.