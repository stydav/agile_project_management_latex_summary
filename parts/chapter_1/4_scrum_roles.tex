\section{scrum roles}

Scrum implements this iterative, incremental skeleton through three roles.

All management responsibilities in a project are divided among these three roles.

\begin{enumerate}
  \item Product Owner
  \item Scrum Master
  \item Team
\end{enumerate}


\subsection{Product Owner}

\begin{itemize}
  \item achieves initial and ongoing funding for the project
  \item responsible for representing the interests of everyone with a stake in the project and its resulting system.
  \item responsible for using the Product Backlog (glossary), for its contents, prioritization, and availability, so that valuable functionality is produced first.
  \item responsible to those funding the project for delivering the vision in a manner that maximizes their ROI, he formulates a plan for doing so that includes a Product Backlog
  \item The Product Owner uses the Product Backlog to give the highest priority to the requirements that are of highest value to the business
\end{itemize}


\subsection{Scrum Master}
\begin{itemize}
  \item traditionally 'project manager', but insted of managing the work he manages the scrum process
  \item responsible for the Scrum process, for teaching Scrum to everyone involved in the project
  \item knows how to guide a Scrum project through the shoals of complexity
  \item implementing Scrum so that it fits within an organization’s culture and still delivers desired benefits
  \item ensuring that everyone follows Scrum rules and practices
  \item helps increase the probability of success by helping the Product Owner select the most valuable Product Backlog and by helping the Team turn that backlog into functionality.
  \item the Team’s welfare is the ScrumMaster’s highest responsibility, he would do anything in his or her power to  help the team be productive
  \item ScrumMasters have to make a personal commitment to their teams, they have to be present at the meetings
  \item the ScrumMaster's role is one without authority
  \item the ScrumMaster is a leader, not a manager
  \item responsible for removing any barriers between the development Teams and the Product Owner and customers so that the customers can directly drive development.
  \item responsible for showing the Product Owner how to use Scrum to maximize project return on investment  (ROI) and meet the project’s objectives
  \item more in \ref{sec:the_scrum_master}
\end{itemize}

The people who fill these roles are those who have committed to the project.

After the Sprint review and prior to the next Sprint planning meeting, the ScrumMaster holds a Sprint retrospective meeting with the Team


\subsection{Team}

\begin{itemize}
  \item responsible for developing functionality
  \item self-managing, self-organizing, and cross-functional
  \item Optimally, a team should include seven people
  \item responsible for figuring out how to turn Product Backlog into an increment of functionality within an iteration
  \item collectively responsible for the success of each iteration and of the project as a whole
  \item compiles an initial list of sprint backlog tasks in the second part of the Sprint planning meeting from the product backlog tasks
  \item Teams handle their own management, selects the work that it will do during each Sprint
  \item After that initial selection is made, it is up to the team to figure out how to do the work at hand.
  \item The team devises its own tasks and figures out who will do them.
  \item has full authority to manage itself to meet the Sprint goal within the guidelines, standards, and conventions of the organization and of Scrum.
  \item Scrum teams are cross-functional
\end{itemize}

Team tells the Product Owner how much of what is desired it believes it can turn into functionality over the next Sprint.


\subsection{two groups}
important distinction in scrum
\begin{itemize}
  \item 'chickens': are intereseted in the project, but not on the hook
  \item 'pigs': are on the hook
\end{itemize}

It should always be clear who is on the hook and who is just a kibitzer.

Who is responsible for the ROI, and who has a stake in the ROI but isn’t accountable?

The rules of Scrum distinguish between the chickens and the pigs to increase productivity, create momentum, and put an end to floundering. 