\section{scrum roles}

Scrum implements this iterative, incremental skeleton through three roles.

All management responsibilities in a project are divided among these three roles.

\begin{enumerate}
  \item Product Owner
  \item Scrum Master
  \item Team
\end{enumerate}


\subsection{Product Owner}

\begin{itemize}
  \item achieves initial and ongoing funding for the project
  \item responsible for representing the interests of everyone with a stake in the project and its resulting system.
  \item responsible for using the Product Backlog (glossary) so that valuable functionality is produced first.
  \item responsible to those funding the project for delivering the vision in a manner that maximizes their ROI, he formulates a plan for doing so that includes a Product Backlog
\end{itemize}


\subsection{Scrum Master}
\begin{itemize}
  \item responsible for the Scrum process, for teaching Scrum to everyone involved in the project
  \item implementing Scrum so that it fits within an organization’s culture and still delivers desired benefits
  \item ensuring that everyone follows Scrum rules and practices
\end{itemize}

The people who fill these roles are those who have committed to the project.

After the Sprint review and prior to the next Sprint planning meeting, the ScrumMaster holds a Sprint retrospective meeting with the Team


\subsection{Team}

\begin{itemize}
  \item responsible for developing functionality
  \item self-managing, self-organizing, and cross-functional
  \item responsible for figuring out how to turn Product Backlog into an increment of functionality within an iteration
  \item collectively responsible for the success of each iteration and of the project as a whole
\end{itemize}

Team tells the Product Owner how much of what is desired it believes it can turn into functionality over the next Sprint.


\subsection{two groups}
important distinction in scrum
\begin{itemize}
  \item 'chickens': are intereseted in the project, but not on the hook
  \item 'pigs': are on the hook
\end{itemize}

It should always be clear who is on the hook and who is just a kibitzer.

Who is responsible for the ROI, and who has a stake in the ROI but isn’t accountable?

The rules of Scrum distinguish between the chickens and the pigs to increase productivity, create momentum, and put an end to floundering. 