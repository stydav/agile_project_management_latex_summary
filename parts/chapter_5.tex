\pagebreak
\chapter{the product owner}


\begin{itemize}
  \item face-to-face communication over documentation
  \item collaboration over reports
\end{itemize}

Downplaying Scrum can simplify collaboration by reducing the overhead of learning something new and the apprehension of a strange new methodology. A Product Backlog could just be a prioritized list of requirements and a month the time between meetings.


A key element in scrum is that Team and the Product Owner need to learn to understand each other.
The Product Owner learns to talk in terms of business requirements and objectives, whereas the Team learns to speak in terms of technology. Because the Product Owner is unlikely to learn the technology, one of the main jobs of the ScrumMaster is to teach the Team to talk in terms of business needs and objectives.

The common denominator between the Team and the Product Owner is the Product Backlog. 

The ScrumMaster can get the Product Owner and the Team to speak the same language, to use a meaningful common vocabulary to discuss a mutual problem. He needs to understand the depths of the language divide that potentially separates customers and developers. Bridging this gap is critical; if both sides can't speak the same language, collaboration can't and won't occur.

The Product Owner has no interest in bridging the gap, and doesn't have the background to do so anyway, so it is up to the ScrumMaster to help the Team bridge the gap.

