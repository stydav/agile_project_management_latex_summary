\pagebreak
\chapter{bringing order from chaos}

keywords: 
\begin{itemize}
  \item time-boxing
  \item incremental delivery
  \item practice of empowerment
  \item self-organization
\end{itemize}

Problems in many organizations: The people who analyzed the situation and set requirements were not the same people who would design the solutions that met these requirements. The people who designed these solutions were not the same people who would code the solutions. 

It can be hard trying to figure out every thing in advance in a complex project. The developers dont have enough time both to build functionality and to debug it. By focusing on \textbf{increments of functionality}, the team makes orderly  progress toward completing the release. Since each increment is tested as it is coded, the number of bugs never overwhelms the project. Scrum’s requirement that each increment of code be potentially shippable requires the  incremental removal of defects and minimizes ongoing bugs.

When using Scrum, teams are empowered to find their own way through \textit{complex situations}.

A Scrum of Scrums (glossary) is the usual mechanism that coordinates multiple teams working on a single project, much as the Daily Scrum is the mechanism that coordinates the work of multiple people on a single team. The work of many teams is coordinated by individuals from each of the teams.

Before a project officially begins, the planners of the project parse the work among teams to minimize dependencies.Teams then work on parts of the project architecture that are orthogonal to each other. This coordination mechanism is effective only when there are minor couplings or dependencies that require resolution, otherwise this wont work.

Sometimes projects are so complex that they require something more than the normal implementation of Scrum. As the degree of complexity rises, the number of inspections must be increased. Because of the increased frequency of inspections, the opportunity for adaptation also increases. The usual Daily Scrum wouldn’t offer enough opportunities for inspection of progress and detection of dependencies at play, and inspection is required for the necessary adaptations to be selected and implemented.

Out-of-the-box Scrum doesn’t have practices that address the complexities of every project. However, ScrumMasters have only to refer back to Scrum theory to find Scrum practices that can be readily adapted to handle even the most complex projects.

For complex projects
\begin{itemize}
  \item reduce the complexity to a degree where the team can cope and function
  \item focus on the next 30 calendar days
  \item forget the rest of the release and to focus on a few concrete steps, the rest falls into place
  \item staff the teams so that all the expertise necessary to develop a piece of functionality was included within each team. Each team will be able to resolve any dependencies it had on other teams. Most members of each team were able to focus on the work at hand, while the cross-team members spent time synchronizing team progress with that of other dependent teams.
\end{itemize}

This enables the team to focus and implement the foundation upon which the rest of the release and future Sprints depend.